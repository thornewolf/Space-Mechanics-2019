HW6.m
\begin{lstlisting}[frame=lines,style=Matlab-editor,basicstyle = \mlttfamily]
clc; close all; clear all

MU = 398600;
a = 8861; % km
AOP = 120; % deg
RAAN = 75; % deg
e = 0.2;
inc = 42; % deg
AOL1 = 60; % deg

true_a1 = AOL1-AOP; %deg
true_a2 = 280; % deg

syms E
eq = tand(true_a1/2) == sqrt((1+e)/(1-e))*tan(E/2);
E1 = double(solve(eq,E));
eq = tand(true_a2/2) == sqrt((1+e)/(1-e))*tan(E/2);
E2 = double(solve(eq,E));

syms t1 t2
eq = sqrt(MU/a^3)*t1 == E1-e*sin(E1);
t1 = double(solve(eq,t1));
eq = sqrt(MU/a^3)*t2 == E2-e*sin(E2);
t2 = double(solve(eq,t2));

syms dt
period = 2*pi*sqrt(a^3/MU);
dt = t2 - t1 + period;
dt_days = dt/60/60/24;

syms theta_era
%
tu = juliandate(datetime('2019-03-20 12:00:00')) - 2451545 + dt_days;
theta_era = 2*pi*(0.7790572732640 + 1.00273781191135448*tu);
theta_era = mod(theta_era,2*pi)*180/pi;


syms lat
eq = sind(lat) == sind(inc)*sind(AOP+true_a2);
lat = double(solve(eq, lat));
lat = lat(1); %Lat is only defined from -90 to 90

syms long
alpha = atan2d(cosd(inc)*sind(AOP+true_a2),cosd(AOP+true_a2));
eq = long == alpha + RAAN - theta_era;
long = double(solve(eq, long)); % Long will always be correct

syms r2
lat_gs = 34.54;
long_gs = -112.4685;
z_gs = 1.62;
AOL2 = AOP + true_a2;
p = a*(1-e^2);
r2 = p/(1+e*cosd(true_a2));
vr2 = [r2 0 0]';
rth_eci = rot_rth_eci(RAAN, inc, AOL2);
eci_ecef = rot_eci_ecef(theta_era);
ecef_sez = rot_ecef_sez(long_gs, lat_gs);
vr2_sez = ecef_sez*eci_ecef*rth_eci*vr2;
vr_e_gs = [0 0 z_gs+6378]';
r_gs_sc = vr2_sez - vr_e_gs;

azimuth = atan2d(r_gs_sc(2),r_gs_sc(1));
elevation = asind(r_gs_sc(3)/norm(r_gs_sc));

% Problem 4
true_a_array = linspace(true_a1, true_a2, 1000);
theta_array = AOP + true_a_array;
lat_array = asind(sind(inc)*sind(theta_array));

alpha_array = atan2d(cosd(inc)*sind(theta_array),cosd(theta_array));
long_array = alpha_array + RAAN - theta_era;

figure; hold on;
geoshow("landareas.shp","FaceColor",[0.5 1.0 0.5])
geoshow(lat_array, long_array,'Color','red')
geoshow(lat_array(1),long_array(1),'DisplayType','Point','Marker','x','Markersize',20)

% GMAT LAT LONG
% 25.55875554722227 76.41433137574843



function A = rot_rth_eci(o,i,t)

% o : Omega, Longitude of the Ascending Node (RAAN)
% i : i, inclination
% t : theta, Argument of Latitude

A = [cosd(o)*cosd(t)-sind(o)*cosd(i)*sind(t), -cosd(o)*sind(t)-sind(o)*cosd(i)*cosd(t), sind(o)*sind(i); ...
        sind(o)*cosd(t)+cosd(o)*cosd(i)*sind(t), -sind(o)*sind(t)+cosd(o)*cosd(i)*cosd(t), -cosd(o)*sind(i); ...
        sind(i)*sind(t), sind(i)*cosd(t), cosd(i)];

end

function A = rot_eci_ecef(t)

% t : theta_era, Earth rotation angle

A = [cosd(t), sind(t), 0; -sind(t), cosd(t), 0; 0, 0, 1];
end

function A = rot_ecef_sez(l,p)

% l : lamda_gs, longitude of the ground station
% p : phi_gs, latitude of the ground station

A = [sind(p)*cosd(l), sind(p)*sind(l), -cosd(p);
    -sind(l), cosd(l), 0;
    cosd(p)*cosd(l), cosd(p)*sind(l), sind(p)];

end

\end{lstlisting}
