\documentclass[hidelinks,12pt]{article}

\usepackage{amsmath}    % need for subequations
\usepackage{graphicx}   % need for figures
\usepackage{verbatim}   % useful for program listings
\usepackage{color}      % use if color is used in text
\usepackage{subfigure}  % use for side-by-side figures
\usepackage{hyperref}   % use for hypertext links, including those to external documents and URLs
\usepackage[numbered]{matlab-prettifier} % including matlab w/ syntax highlighting
\usepackage[T1]{fontenc} % prettier matlab font
\lstMakeShortInline[style=Matlab-editor]| % matlab inline escape character
\graphicspath{ {./Figures/} }

\usepackage[
top    = 2.75cm,
bottom = 2.50cm,
left   = 3.00cm,
right  = 2.50cm]{geometry}



% don't need the following. simply use defaults
\setlength{\baselineskip}{16.0pt}    % 16 pt usual spacing between lines



\begin{document}
\pagenumbering{gobble}
\begin{center}
  {\huge Homework 4}\\
  \vspace{10px}
  \includegraphics{Logo} \\
  Date of Submission:\\
  February 8, 2019\\
  \vspace{30px}
  \rule{300px}{0.5px} \\
  Thorne Wolfenbarger \\
  \href{mailto:wolfent1@my.erau.edu}{wolfent1@my.erau.edu} \\
  \vspace{30px}
  Submitted to: \\
  Professor Karla Martin \\
  College of Engineering \\
  \vspace{40px}
  In Partial Fulfillment \\
  of the Requirements of \\
  \vspace{10px}
  AE 313 \\
  Space Mechanics \\
  Spring, 2019 \\
\end{center}

\pagenumbering{roman}
{\tableofcontents\let\clearpage\relax}
\clearpage
\pagenumbering{arabic}
\newpage
\begin{flushleft}

\section{Calculate the center of mass with respect to the sun}
Using the center of mass equation (Eq. \ref{eq:COM}) we produced MATLAB code to solve for the system's center of mass with respect to the sun.
\begin{equation} \label{eq:COM}
\vec{r}_{cm} = \frac{ \Sigma_{i=1}^{n}m_i \vec{r}_{Sun \to i} }{\Sigma_{i=1}^{n}m_i}
\end{equation}
\vspace{5px}
\begin{lstlisting}[frame=lines,style=Matlab-editor,basicstyle = \mlttfamily]
rCM = (rSunMars * M_Mars + rSunPhobos * M_Phobos + rSunInSight * M_InSight)/(M_InSight + M_Phobos + M_Mars + M_Sun);
\end{lstlisting}
\vspace{5px}
The center of mass with respect to the sun is $(64.1296 \hat{\imath} + 23.5509 \hat{\jmath} -0.1080\hat{k})~km$

\section{Find the acceleration of InSight}
\begin{large}
  Find the acceleration of InSight ($\ddot{\vec{r}}_{InSight}$). What is the acceleration relative to?
\end{large}\\
\vspace{5px}
Using (Eq. \ref{eq:ACC}) and MATLAB we calculated $\ddot{\vec{r}}_{InSight}$
\begin{equation} \label{eq:ACC}
\ddot{\vec{r}}_i = - \Sigma_{j=1}^{n} \frac{Gm_j}{r_{j \to i}^3} \cdot \vec{r}_{j \to i}
\end{equation}
\vspace{5px}
\begin{lstlisting}[frame=lines,style=Matlab-editor,basicstyle = \mlttfamily]
Accel_InSight = -G * (((M_Sun * rSunInSight) / norm(rSunInSight)^3 )+((M_Phobos * rPhobosInSight) / (norm(rPhobosInSight)^3 )) +((M_Mars * rMarsInSight)/ (norm(rMarsInSight)^3)));
\end{lstlisting}
\vspace{5px}
The acceleration of InSight is $(-2.744 \hat{\imath} -1.116 \hat{\jmath} + 0.007\hat{k}) \cdot 10^{-6}~km/s^2$. This acceleration is relative to an inertial point, which is the center of mass of the system. (Calculated in Eq. \ref{eq:COM})
\section{Write the relative motion differential equation}
\begin{large}
  Write the differential equation that governs the relative motion of InSight relative to Mars.
\end{large}
\begin{multline}
  \ddot{\vec{r}}_{Mars \to InSight} = \frac{-G(m_{InSight} + m_{Mars})}{r^3_{Mars \to InSight}} \cdot \vec{r}_{Mars \to InSight} + G \cdot m_{Sun}(\frac{\vec{r}_{InSight \to Sun}}{r^3_{InSight \to Sun}} - \frac{\vec{r}_{Mars \to Sun}}{r^3_{Mars \to Sun}}) \\
  + G \cdot m_{Phobos}(\frac{\vec{r}_{InSight \to Phobos}}{r^3_{InSight \to Phobos}} - \frac{\vec{r}_{Mars \to Phobos}}{r^3_{Mars \to Phobos}})
\end{multline}

\section{Determine Accelerations}
\begin{large}
  Determine the dominant, direct, and indirect acceleration (vector) on InSight.
\end{large}\\
\vspace{5px}
Dominant acceleration on InSight:
\begin{equation} \label{eq:DomAcc}
  \frac{-G(m_{InSight} + m_{Mars})}{r^3_{Mars \to InSight}} \cdot \vec{r}_{Mars \to InSight} = (3.206 \hat{\imath} -8.66 \hat{\jmath} + 0.243 \hat{k}) \cdot 10^{-8}~km/s^2
\end{equation}
Direct accelerations on InSight:\\
~~~~Due to the Sun:\\
\begin{equation}
  G \cdot m_{Sun} \cdot \frac{\vec{r}_{InSight \to Sun}}{r^3_{InSight \to Sun}}
  = (-2.776 \hat{\imath} -1.030 \hat{\jmath} + 0.005\hat{k}) \cdot 10^{-6}~km/s^2
\end{equation}
~~~~Due to Phobos:
\begin{equation}
  G \cdot m_{Phobos} \cdot \frac{\vec{r}_{InSight \to Phobos}}{r^3_{InSight \to Phobos}} = (0.511 \hat{\imath} -1.401 \hat{\jmath} + 0.037\hat{k}) \cdot 10^{-15}~km/s^2
\end{equation}
Indirect accelerations on InSight:\\
~~~~Due to the Sun:\\
  \begin{equation}
    -G \cdot m_{Sun} \cdot \frac{\vec{r}_{Mars \to Sun}}{r^3_{Mars \to Sun}}
    = (2.780 \hat{\imath} + 1.021 \hat{\jmath} - 0.005\hat{k}) \cdot 10^{-6}~km/s^2
  \end{equation}
~~~~Due to Phobos:
  \begin{equation} \label{eq:IndirPhobos}
    -G \cdot m_{Phobos} \cdot \frac{\vec{r}_{Mars \to Phobos}}{r^3_{Mars \to Phobos}} = (-0.151 \hat{\imath} + 8.089 \hat{\jmath} + 0.609\hat{k}) \cdot 10^{-12}~km/s^2
  \end{equation}
\section{Evaluate Accelerations}
\begin{large}
  For the same relative acceleration (InSight relative to Mars), evaluate the acceleration due to Mars, acceleration due to the Sun, and acceleration due to Phobos.
\end{large}\\
\vspace{5px}
Using equations \ref{eq:DomAcc}-\ref{eq:IndirPhobos} and MATLAB we solve for the various accelerations induced on InSight by other celestial bodies.
\vspace{5px}
\begin{lstlisting}[frame=lines,style=Matlab-editor,basicstyle = \mlttfamily]
Accel_InSight_due_to_Mars = (-G * (M_InSight + M_Mars) / (norm(rMarsInSight)^3) * (rMarsInSight));
Accel_InSight_due_to_Sun = (G * (M_Sun * ((rInSightSun / norm(rInSightSun)^3)- (rMarsSun / norm(rMarsSun)^3))));
Accel_InSight_due_to_Phobos = (G * (M_Phobos * ((rInSightPhobos / norm(rInSightPhobos)^3)- (rMarsPhobos/norm(rMarsPhobos)^3))));
\end{lstlisting}
\vspace{5px}
Acceleration of InSight due to Mars is $(3.206 \hat{\imath} -8.666 \hat{\jmath} + 0.243\hat{k}) \cdot 10^{-8}~km/s^2$\\
Acceleration of InSight due to the Sun is $(3.282 \hat{\imath} -8.935 \hat{\jmath} + 0.251 \hat{k}) \cdot 10^{-9}~km/s^2$\\
Acceleration of InSight due to Phobos is $(-1.508 \hat{\imath} + 0.8088 \hat{\jmath} + 0.0695 \hat{k}) \cdot 10^{-13}~km/s^2$
\section{Is it Reasonable?}
\begin{large}
  At this instant is it reasonable to assume a two-body relative model for motion of the spacecraft with respect to Mars? Why or why not?
\end{large}\\
\vspace{5px}
It is not reasonable to assume a 2-body relative model for motion of the spacecraft with respect to Mars. The gravitational influence of the Sun on InSight is approximately 10\% of the influence of Mars on InSight. If a two-body relative model of motion was assumed between InSight and Mars, 10\% of the total accelerative forces would be ignored. This opens up potential for mission disrupting inaccuracies.

\end{flushleft}
\newpage
HW5.m
\begin{lstlisting}[frame=lines,style=Matlab-editor,basicstyle = \mlttfamily]
clc; clear all; close all

%constants
MU = 42828;

% Problem 1
vr = [-2089.6 -2515.7 -6382.0];
vv = [2.1744 0.76911 0.13452];

r = norm(vr)
v = norm(vv)

vh = cross(vr, vv);
h = norm(vh);

syms a
energy_eqn = v^2/2 - MU/r == -MU/(2*a)
energy = v^2/2 - MU/r
a = double(solve(energy_eqn, a))

syms e
h_eqn = h == sqrt(MU*a*(1-e^2))
e = double(solve(h_eqn,e))
e = max(e)

syms E
E_eqn = r == a*(1-e*cos(E));
E = double(solve(E_eqn,E));
E = -min(E) %Because it is descending.

syms time_since_p
time_since_p_eqn = sqrt(MU/a^3)*time_since_p == E - e*sin(E);
time_since_p = double(solve(time_since_p_eqn,time_since_p))

% Problem 2
p = h^2/MU;
true_a = -acos(p/(r*e) - 1/e)
vv_rt = [h*e/p*sin(true_a) h/r 0] % answer

% Problem 3
E1 = E;
t1 = time_since_p;
vr1 = vr;
vv1 = vv;
r1 = r;
v1 = v;
t2 = time_since_p + 12*60*60;
n = sqrt(MU/a^3);
M = n*t2;
E2 = fzero(@(x) x-e*sin(x)-M, 0);
f = 1 - a/r*(1-cos(E2-E1));
g = (t2 - t1) - sqrt(a^3/MU)*(E2 - E1 - sin(E2 - E1));

vr2 = f*vr + g*vv;
r2 = norm(vr2);
fdot = -sqrt(MU*a)/(r2*r1)*sin(E2-E1);
gdot = 1 - a/r2*(1-cos(E2-E1));

vv2 = fdot*vr1 + gdot*vv1 % intertial unit vectors

true_a2 = -acos(p/(r2*e) - 1/e)

vv2_rt = [h*e/p*sin(true_a2) h/r2 0] % radial-tangential

% Problem 4
vr2_peri = [r2*cos(true_a2) r2*sin(true_a2) 0]
vv2_peri = sqrt(MU/p)*[-sin(true_a2) e+cos(true_a2) 0]

% Problem 5

% find i theta omega
syms inc
h_hat = cross(vr2, vv2)/norm(cross(vr2, vv2))
inc_eqn = cos(inc) == dot(h_hat, [0 0 1])
inc = double(solve(inc_eqn,inc))
inc = max(inc)

syms RAAN
RAAN_eqn_1 = sin(RAAN)*sin(inc) == dot(h_hat, [1 0 0])
RAAN_eqn_2 = -cos(RAAN)*sin(inc) == dot(h_hat, [0 1 0])
RAAN1 = double(solve(RAAN_eqn_1));
RAAN2 = double(solve(RAAN_eqn_2));
RAAN = min(RAAN1)*180/pi

syms arg_peri
r1_hat = vr1/norm(vr1)
theta_hat = cross(r1_hat, h_hat)/norm(cross(r1_hat, h_hat))
syms theta
theta_eqn_1 = sin(inc) * sin(theta) == dot(r1_hat, [0 0 1])
theta_eqn_2 = sin(inc) * cos(theta) == dot(theta_hat, [0 0 1])
theta1 = double(solve(theta_eqn_1, theta))
theta2 = double(solve(theta_eqn_2, theta))
theta = intersect(theta1, theta2)
arg_peri_eqn = arg_peri == theta - true_a
arg_peri = double(solve(arg_peri_eqn, arg_peri))
\end{lstlisting}

\end{document}
