HW9.m
\begin{lstlisting}[frame=lines,style=Matlab-editor,basicstyle = \mlttfamily]
clc; clear; close all
% Constants
planets = {'Sun', 'Moon', 'Mercury', 'Venus', 'Earth', 'Mars', 'Jupiter', 'Saturn', 'Uranus', 'Neptine', 'Pluto'};
rad_list = [695990.0, 1739.2, 2439.7, 6051.9, 6378.0, 3397.0, 71492.0, 60268.0, 25559.0, 25269.0, 1162.0];
mu_list = [132712440000.0, 4902.8, 22032.0, 324860.0, 398600.0, 42828.0, 126713000.0, 37941000.0, 5794500.0, 6836500.0, 981.6];
sma_list = [0.0, 384400.0, 57910000.0, 108210000.0, 149600000.0, 227920000.0, 778570000.0, 1433530000.0, 2872460000.0, 4495060000.0, 5906380000.0];
R = containers.Map(planets,rad_list);
MU = containers.Map(planets,mu_list);
r = containers.Map(planets,sma_list);
%%%%%%%

STEPSIZE = 0.0001;

ta_range = [-61.2170  110.0000];
step_count = (ta_range(2) - ta_range(1))/STEPSIZE;
% 1.
ta_list = linspace(ta_range(1)+0.01, ta_range(2), step_count);
r1 = R('Earth')*1.7;
r2 = R('Earth') + 20460;
c = 35828.32;
TA = 140;
emin = (r2 - r1)/c;
xi = atan2d( (r1/r2 - cosd(TA)),sind(TA) );

et  = emin ./ sind(ta_list + xi);

pt = r1 .* (1 + et.*cosd(ta_list));

at = pt ./ (1 - et.^2);

% conditions on transfer arc at departure
r1p = r1;
v1p = sqrt(2*MU('Earth')/r1p - MU('Earth')./at);

FPA1p = atan2d(r1p .* et .* sind(ta_list), pt);

% find dv and alpha1p
vc = sqrt(MU('Earth') / r1);
v1m = vc;
dv1 = sqrt( v1m^2 + v1p.^2 - 2*v1m.*v1p.*cosd(FPA1p) );

nu = acosd( (v1p.^2 - v1m.^2 - dv1.^2)./(-2.*v1m.*dv1) );

alpha = 180 - nu;

% find conditions at arrival (v2m r2m FPA2m)
r2m = r2;
v2m = sqrt(2*MU('Earth')/r2m - MU('Earth')./at);

ta2m = ta_list + TA;
FPA2p = atan2d(r2m * et .* sind(ta2m), pt);

vc = sqrt(MU('Earth')/r2);
v2p = vc;

dv2 = sqrt( v2m.^2 + v2p.^2 - 2.*v2m.*v2p.*cosd(FPA2p) );

nu = acosd( (v2p.^2 - v2m.^2 - dv2.^2)./(-2.*v2m.*dv2) );

alpha = 180 - nu;

dvt = dv1 + dv2;

figure(1)
hold on
% plot(ta_list, dvt)
plot(FPA1p, ta_list, 'DisplayName', '\gamma_1')
plot(FPA2p, ta_list, 'DisplayName', '\gamma_2')

[mindv ind] = min(dvt);
plot(FPA1p(ind), ta_list(ind), 'r.', 'HandleVisibility','off')
plot(FPA2p(ind), ta_list(ind), 'r.', 'DisplayName', 'Minimum \Delta v')

xlabel('$\theta^*$ (deg)','Interpreter','latex')
ylabel('$|\Delta \vec{v}|~/~\gamma_1~/~\gamma_2$','Interpreter','latex')
legend

% 4. what type of transfer
figure(2)
ta_min_dep = ta_list(ind);
ta_min_arr = ta2m(ind);
et_min = et(ind);
at_min = at(ind);
t = linspace(ta_min_dep-10, ta_min_arr-10, 100);
major = at_min;
minor = at_min*sqrt(1-et_min^2);
x = major*cosd(t);
y = minor*sind(t);
focus = sqrt(major^2 - minor^2);
x = x-focus;
hold on
plot(x,y, 'linestyle', '--');
% plot([0 major*cosd(ta_min_dep)-focus], [0 minor*sind(ta_min_dep)], 'color', 'red');
% plot([0 major*cosd(ta_min_arr)-focus], [0 minor*sind(ta_min_arr)], 'color', 'red');
energy_min = MU('Earth')/at(ind);
% 4.
% Transfer type is a 1A transfer. Ellipse with TA < 180 & focus not between
%chord and arc
t = linspace(0, 2*pi, 100);
x = r1*cos(t);
y = r1*sin(t);
plot(x,y)

x = r2*cos(t);
y = r2*sin(t);
plot(x,y)
axis equal
title('Initial, Final, and Transfer Orbits')
set(gca,'XTick',[], 'YTick', [])
legend('Transfer Orbit', 'Orbit 1', 'Orbit 2')

% 6.
dv_compared = 3.62 - dvt(ind);
% The minimum delta v transfer orbit saves 1.0173 km/s of delta v compared
% to the minimum energy orbit. The semi-major axis of the min-dv transfer
% orbit is larget than the semi-major axis of the minimum energy orbit.
% This indicated that the min-dv transfer will have a longer TOF.

\end{lstlisting}
